\chapter{Introduction}
\label{chap:Intro}

\begin{quotation}

\vspace{-3cm}

\begin{flushright}
\begin{minipage}[t][5cm][b]{0.5\textwidth}
{\letquote ``Wizards don’t like philosophy very much. As far as they are concerned, one hand clapping makes a noise like `cl'.''}

\bigskip

-{\small  Terry Pratchett, Sourcery }
\end{minipage}
\end{flushright}

\vspace{0.5cm}

\end{quotation}




Understanding how cooperation emerges and evolves in complex social systems remains one of the most fascinating challenges across multiple disciplines. This thesis explores this challenge through three interconnected perspectives: evolutionary game theory, complexity analysis, and strategic control.



We begin by examining how cooperation can emerge in the public goods game when faced with temporal variations.
%TODO Explain a little

 While previous research has typically assumed static conditions, we investigate how oscillating enhancement factors and delayed punishment affect the evolution of cooperative behavior. This approach provides novel insights into more realistic scenarios where the benefits of cooperation and the consequences of actions fluctuate over time.

%TODO Main results

Our investigation then shifts to analyzing the complexity of spatial patterns in evolutionary games. By applying the Hamming distance metric to both the prisoner's dilemma and public goods games, we develop new methods for quantifying and characterizing the complexity of a system. This analysis reveals how minimal variations, through local interactions can lead to enormous global changes and helps identify parameter regions where complex dynamics emerge.

Traditionally, %TODO Lyapunov exponents


%TODO Main results

Then, we shift our attention to cellular automata, as a more simple case of evolutionary game. This study provides another perspective on pattern formation. Through a systematic analysis of elementary cellular automata, we explore how simple local rules can generate complex global behaviors. Furthermore we establish a new classification of the elementary cellular automata that helps understand better the fundamental classification of Wolfram for cellular automata \cite{WolframClass}.

%TODO Main results




Finally, we extend the concept of partial control to competitive scenarios, developing a novel framework for analyzing strategic interactions in chaotic systems. 

%TODO Partial Control

In the study of Chapter~\ref{chap:ForcingEscape} we begin by extending the study of partial control to a case in which the controller aims to expell the trajectory from the chaotic region, instead of remaining there. %TODO

The study culminates in Chapter~\ref{chap:PartialControlGame} with a two-player game where participants compete for control over a chaotic trajectory, revealing how information asymmetry and timing of actions influence strategic outcomes.

%TODO Main results


Our findings contribute to a deeper understanding of how game theory, complexity, and control interplay in social and dynamical systems, with potential applications ranging from social physics to game theory and chaos control.










\begin{thebibliography}{01}





\end{thebibliography}

