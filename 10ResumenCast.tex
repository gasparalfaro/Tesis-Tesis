\chapter*{Resumen de la tesis en castellano}

\addcontentsline{toc}{chapter}{Resumen de la tesis en castellano}  
\markboth{Resumen de la tesis en castellano}{}

\begin{quotation}
	\vspace{-3cm}
    \begin{flushright}
    \begin{minipage}[t][5cm][b]{0.5\textwidth}
    {\letquote ``Que la realidad te parezca absurda es la mayor crítica que le puedes hacer"}
    
    \bigskip
    
    -{\small  Laura Fernández}
    \end{minipage}
    \end{flushright}
    
    \vspace{0.5cm}
\end{quotation}




\section{Introducción}

Esta tesis trata sobre juegos evolutivos, complejidad y control del caos. Estos temas, que pueden parecer dispares, han sido interconectados en los diferentes capítulos presentes aquí. El hilo conductor entre las distintas disciplinas es que todos han sido estudiados mediante herramientas propias del análisis de los sistemas complejos y la dinámica no lineal. 

La Teoría de Juegos Evolutiva (Evoluytionary Game Theory, EGT) es analizada profundamente en ramas de la física por su dinámica compleja. Al ser un estudio más adecuado para grandes poblaciones, los sistemas estudiados con estas técnicas suelen tener propiedades emergentes. Estas propiedades emergentes no pueden ser explicadas considerando únicamente a los individuos que forman el sistema, sino que hay que tratar con la población entera para que aparezcan y poder entenderlos. Como consecuencia, estos sistemas suelen tener propiedades no lineales muy interesantes para la física de sistemas complejos.

Al principio de la tesis se ha estudiado el juego de los bienes públicos. En este juego social, los jugadores tratan de hacer grupos con los que invertir para aumentar sus recompensas. Cada individuo puede elegir cooperar con una unidad monetaria al bien público o abstenerse. Indiferentemente a si han cooperado o no, todos los jugadores se reparten el resultado de la inversión, con su debido aumento gracias a las ganancias, al final de cada ronda. Según la estrategia que adapta cada individuo y la que adaptan sus vecinos, este obtiene un \textit{payoff}, es decir, un saldo que determina la aptitud o \textit{fitness} del individuo en el juego. Mediante un proceso evolutivo consistente en la imitación del individuo más apto, el sistema evoluciona según se iteran turnos del juego.

Aparte de las dos estrategias ya mencionadas, cooperar o no hacerlo, se ha añadido una tercera estrategia. Existen individuos que, aparte de cooperar, pagan una tasa adicional para castigar a los individuos que no cooperan. Los impagadores que sean detectados por estos castigadores deberán pagar una multa, reduciendo así su \textit{payoff}.

En la investigación del primer capítulo se ha analizado la variación en el juego de los bienes públicos al introducir componentes dependientes del tiempo. Primero se ha considerado una oscilación en el parámetro que controla la eficacia de las inversiones. Esta oscilación trata de imitar las fluctuaciones propias de los mercados o de la naturaleza. Se ha observado que esta oscilación afectaba negativamente a la cooperación entre individuos. 

También se ha investigado el efecto de introducir un retardo a la hora de castigar a los impagadores. Este retardo causa que los impagadores aumenten, demostrando así que los castigos más eficaces son aquellos que son inmediatos.

Tras haber resuelto estas cuestiones nos propusimos analizar la complejidad del juego, además de otro juego social, el dilema del prisionero. Otros estudios demuestran que se producen comportamientos caóticos espacio-temporales considerando la dinámica de formación de patrones cuando se observa la disposición de cooperadores y no cooperadores en el espacio. 

Con la ambición de cuantificar esta complejidad, hemos considerado la divergencia de la distancia de Hamming entre dos configuraciones inicialmente muy próximas. Esta distancia, usada comúnmente en ciencias de la computación, establece la diferencia en número de caracteres entre dos grupos del mismo tamaño. Observando su comportamiento hemos podido observar que la distancia crece cuando tratamos con parámetros del sistema en los que existen comportamientos caóticos.

En el caso en el que no se utilice ningún componente aleatorio, los juegos anteriormente estudiados se pueden considerar como autómatas celulares. Eso sí, el número de reglas que tendrían estos autómatas celulares sería muy grande. En concreto el autómata tendría $2^N$ reglas, siendo $N$ el número de vecinos que influyen en el \textit{payoff} de cada agente. Debido a esto, se ha visto conveniente analizar la complejidad de todos los autómatas celulares elementales. Además se ha elaborado una clasificación análoga a la de Wolfram para estos autómatas. Esta clasificación depende del comportamiento de la distancia de Hamming entre dos configuraciones muy próximas al principio.

En otra línea diferente de la investigación se extendió el estudio del control parcial para controlar el escape de una trayectoria de su región caótica transitoria. El control parcial es un método de control en el que se evita ciertas zonas del espacio de fases de un sistema. De esta manera, por ejemplo, en un sistema con dinámicas de caos transitorio, se puede conseguir que la trayectoria permanezca en la región caótica transitoria de manera indefinida. En uno de los estudios realizados en esta tesis se ha considerado exactamente lo contrario. Es decir, hemos tratado de sacar al sistema de esa región controladamente utilizando las mismas herramientas del control parcial.

El sistema naturalmente escapa de la región, pero lo que hemos querido hacer es tratar de hacerlo de manera controlada. Para ello nos propusimos dos objetivos: hacer que la trayectoria escape lo más rápidamente posible, o que escape en un número de iteraciones a elegir por el controlador. 

Tras conseguirlo con éxito nos fijamos un tercer objetivo más ambicioso: hacer que una trayectoria en un sistema multiestable pase de una región a otra de manera controlada. De esta forma, las trayectorias que, sin control transitan de manera caótica de una región a otra, permanecerán en cada región el tiempo que decidamos. 

Por último se ha diseñado un juego de supervivencia entre dos jugadores enfrentados entre sí. Cada jugador tiene un objetivo distinto, permanecer en diferentes zonas del espacio de fases. Hemos ilustrado el juego con un ejemplo, el mapa logístico. Debido a la naturaleza transitoria de este sistema, este mapa se plantea muy interesante. Cuando establecemos que un jugador tiene como objetivo escapar de la región caótica transitoria y el oponente trata de permanecer en esa zona, el jugador que quiere escapar tiene ventaja. Las trayectorias naturalmente escaparán dándole la victoria. Gracias a las técnicas de control parcial, hemos obtenido las condiciones iniciales para las que cada jugador tiene asegurada la victoria, los sets de victoria, \textit{winning sets}. 

Con las herramientas y descubrimientos realizados durante el transcurso de esta tesis doctoral, esperamos haber contribuido a las ramas de los juegos evolutivos, los autómatas celulares y el control del caos, y haber facilitado futuras investigaciones en estos temas.

\section{Metodología}

Para realizar esta tesis hemos empleado un análisis teórico y computacional. Con ayuda de métodos analíticos u simulaciones numéricas, resolvemos y corroboramos predicciones teóricas. A continuación describimos los métodos utilizados.


Se han realizado simulaciones Monte Carlo para modelos basados en agentes. De esta manera tenemos un modelo evolutivo de individuos que juegan a juegos sociales. Hemos utilizado una red cuadrada en la que cada agente está colocado y juega con sus vecinos inmediatos. Según el método de Monte Carlo, se eligen individuos al azar para que adopten la estrategia de un vecino aleatorio. Si el \textit{payoff} del vecino es mayor, es probable que adopte su estrategia. Las simulaciones han sido realizadas en el lenguaje de programación \textit{Julia}.

En un momento de la investigación hemos diseñado un mapa personalizado para aplicarle el método de control parcial. Para caracterizar la dinámica de este nuevo mapa se ha realizado un diagramas de bifurcación, que determina la estabilidad del mapa ante diferentes valores de los parámetros que determinan el sistema, en nuestro caso, un solo parámetro. 

Además se obtuvo el máximo exponente de Lyapunov para el mapa. Este exponente cuantifica la naturaleza caótica de un sistema. Cuando el exponente toma valores positivos se considera que sus órbitas serán caóticas, y cuando el exponente es negativo o nulo, el sistema es estable, caracterizado por órbitas periódicas. El máximo exponente de Lyapunov se puede obtener con la siguiente fórmula cuando el sistema es discreto en el tiempo.
\begin{equation*}
\lambda(x_0) = \lim_{n \to \infty}\dfrac{1}{n}\sum_{i=0}^{n-1}\log\left|\dfrac{df(x_i)}{dt}\right|.
\end{equation*}

El diagrama de bifurcación y el máximo exponente de Lyapunov fueron computados con \textit{MATLAB}.

También se ha aprovechado otra herramienta que permite cuantificar el comportamiento caótico. Este método ha sido escasamente utilizado en la literatura, pero se nos ofrece muy útil en nuestra investigación. Es un algoritmo que utiliza la distancia de Hamming para medir divergencias en los patrones de distribución de diferentes estrategias en juegos o estados de un autómata celular. Se examinaron dos configuraciones inicialmente muy próximas y medimos la separación a medida que el tiempo de la simulación iba aumentando. 


\section{Resultados y Discusión}

En esta tesis hemos realizado aportaciones en los campos de dinámica evolutiva de juegos sociales con la que hemos ganado entendimiento en temas como la cooperación entre individuos. Al hacer oscilar un parámetro que controla la efectividad de inversiones económicas observamos lo siguiente. Cuando el parámetro aumenta hay un aumento de los cooperadores, pero cuando el parámetro disminuye, los cooperadores flaquean y aumenta mucho más el número de impagadores. Este efecto puede ser explicado facilmente, ya que una pobre estabilidad en los rendimientos de una inversión la hace más arriesgada, y, por tanto, menos apetecible. Además se ha comprobado que un castigo tardío no es tan eficaz como un castigo más inmediato.

También hemos hecho progresos relacionados con la cuantificación de la complejidad en sistemas que presentan caos espacio-temporal. Hemos sido capaces de analizar la complejidad en la dinámica de patrones de disposición de diferentes estrategias para dos juegos sociales: el juego de los bienes públicos y el dilema del prisionero. Ambos resultaron presentar complejidad tras medir la divergencia entre configuraciones muy parecidas al principio. Se ha observado que la distancia de Hamming entre las dos configuraciones crece rápidamente para aquellos parámetros en los que se observa comportamiento caótico espacio-temporal.

Además tras el estudio de la distancia de Hamming en autómatas celulares se obtuvo una clasificación análoga a la de Wolfram. Para la Clase-$1$ la distancia era nula y para la Clase-$2$ la distancia permanecía constante, lo cual muestra que los autómatas de ambas clases son sistemas simples. Esto se puede apreciar observando los patrones uniformes que forman los autómatas de estas clases. En cambio, en la Clase-$3$, al mostrar la distancia de Hamming en función del tiempo, esta se comporta de manera caótica. Esto explica los patrones que parecen aleatorios que se observan al representar los estados de estos autómatas celulares. La última clase, la Clase-$4$, también se comporta de manera caótica, pero al cabo de un tiempo alcanza un equilibrio. La distancia tomará un valor fijo o formará una oscilación periódica. Este comportamiento caótico transitorio de la Clase-$4$ se podría relacionar con el fenómeno de \textit{edge of chaos} (límite del caos), y se manifiesta en la combinación de zonas con cierto orden y otras zonas complejas al contemplar los estados del autómata.

La investigación referente a controlar trayectorias para que escapen rápidamente u ordenadamente con control parcial fue exitosa. Se obtuvo el control necesario para expulsar las trayectorias y, para algunas condiciones iniciales, el control resultó ser menor que el ruido propio del sistema. Además se consiguió transformar trayectorias caóticas en quasi-periódicas en sistemas multiestables.

Finalmente se diseño un juego entre dos controladores y fue resuelto mediante técnicas de control parcial. En el juego, uno de los controladores tiene el objetivo de permanecer en una región caótica transitoria y el otro hace todo lo posible por expulsar la trayectoria de allí. Se obtuvieron los sets de victoria, esto es, las condiciones iniciales en las que cada controlador consigue su objetivo.

En el caso de que un jugador conozco los movimientos del rival antes de ejercer su control, los sets de ambos jugadores son complementarios, llenando así todo el espacio de fases. En cambio, si ambos jugadores desconocen las acciones del oponente, puede haber casos en el que queden huecos en los que ningún controlador tenga asegurada la victoria. Es importante resaltar que esto no siempre ocurre, y a veces no hay diferencia entre estar informado y no estarlo. Esto se debe a que las zonas donde el jugador tiene asegurada la victoria son exactamente iguales cuando el controlador está informado que cuando no. Sin embargo, cuando el controlador sí está informado, el valor de control que tiene que hacer es menor.

Un resultado sorprendente es el hecho de que el jugador que está en desventaja, debido a la dinámica natural del mapa con caos transitorio, pude conseguir su objetivo en algunos casos incluso con una cota de control menor que la del oponente.

\subsection{Conclusiones}

Las principales conclusiones del estudio realizado en la tesis doctoral son:

\begin{itemize}

\item En el estudio de los efectos dependientes del tiempo en el juego de los bienes públicos se determinó que la inestabilidad de las inversiones, caracterizada por una oscilación en el parámetro que multiplica las aportaciones de los individuos, perjudica a la cooperación entre estos para invertir conjuntamente. Además, un castigo tardío a los impagadores favorece su crecimiento.

\item  Hemos obtenido una herramienta que cuantifica comportamientos caóticos espacio-temporales de juegos en los que los individuos se reparten de manera espacial. Aunque el sistema esté en equilibrio globalmente, es decir, la frecuencia de poblaciones no se altera significativamente con el tiempo, la dinámica local puede ser inestable. De esta manera cambios mínimos en las configuraciones iniciales de estrategias producen cambios completamente diferentes tras un tiempo característico del sistema, que hemos podido medir.

\item Hemos clasificado correctamente todos los autómatas celulares elementales mediante el uso de la misma herramienta. Las Clases-$1$ y $2$ resultan presentar estabilidad mientras que las Clases-$3$ y $4$ presentan comportamiento caótico. En la Clase-$3$ los autómatas producen caos para siempre mientras que en la Clase-$4$ se ha observado caos transitorio.


\item Hemos elaborado una nueva aplicación al control parcial para controlar el escape de trayectorias de una zona con caos transitorio. Se consiguen expulsar a las trayectorias de la manera más rápida posible o de una manera controlada, sabiendo el tiempo que va a pasar hasta que escape. Además, se ha conseguido crear trayectorias que transitan de una zona a otra periódicamente en sistemas caóticos multiestables.


\item  Usando técnicas de control parcial conseguimos resolver un juego novel de control y supervivencia entre dos jugadores. Se obtuvieron las regiones en las que cada jugador gana el juego dependiendo de la información que cada uno posee.



\end{itemize}
