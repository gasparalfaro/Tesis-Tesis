\chapter*{Curriculum Vitae}

\addcontentsline{toc}{chapter}{Curriculum Vitae}  
\markboth{Curriculum Vitae}{}

\hspace{10cm}\includegraphics[width=0.3\textheight]{Images/picture.jpg}



\section*{Publicaciones}

\begin{itemize}

\item
%\bibitem{Alfaro2021}
G. Alfaro, R. Capeáns, and M. A. F. Sanjuán,
Forcing the escape: Partial control of escaping orbits from a
transient chaotic region,
Nonlinear Dyn. \textbf{104}, 1603--1612 (2021).
\url{https://doi.org/10.1007/s11071-021-06331-4}

\item
%\bibitem{Alfaro2022}
G. Alfaro and M. A. F. Sanjuán,
Time-dependent effects hinder cooperation on the public goods game,
Chaos, Solitons Fractals \textbf{160}, 112206 (2022).
\url{https://doi.org/10.48550/arXiv.2501.12188}


\item
%\bibitem{Alfaro2024}
G. Alfaro and M. A. F. Sanjuán,
Hamming distance as a measure of spatial chaos in evolutionary games,
Phys. Rev. E \textbf{109}, 014203 (2024)
\url{10.1103/PhysRevE.109.014203}

\item
G. Alfaro and M. A. F. Sanjuán,
Classification of cellular automata based on the Hamming
distance,
Chaos \textbf{34}, 083129 (2024)
\url{https://doi.org/10.1063/5.0227349}


\item
%\bibitem{PartialControlGame}
G. Alfaro, R. Capeáns, M. A. F. Sanjuán,
Two-player Yorke's game of survival in chaotic transients,
(2025)
\url{https://doi.org/10.48550/arXiv.2501.12188}


\end{itemize}



\section*{Presentaciones en congresos, seminarios y talleres}

\begin{itemize}

\item
\textbf{Seminario:} Seminarios de Investigación del Grupo de Dinámica No Lineal, Teoría del Caos y Sistemas Complejos.

\textbf{Presentación oral}: Dinámica evolutiva en el juego de los bienes públicos.



\textbf{Lugar y fecha}: Universidad Rey Juan Carlos, 21 de abril de 2022.

\item
\textbf{Taller:} Ciencia a la carta.

\textbf{Presentación oral}: Taller de fractales.

\textbf{Lugar y fecha}: Universidad Rey Juan Carlos, 5-7 de abril de 2022.

\item
\textbf{Taller:} XXII Semana de la Ciencia y la Innovación de Madrid.

\textbf{Presentación oral}: Falsifica un Pollock.

\textbf{Lugar y fecha}: Universidad Rey Juan Carlos, 7-20 de noviembre de 2022.

\item
\textbf{Taller:} XXIII Semana de la Ciencia y la Innovación de Madrid.

\textbf{Presentación oral}: Arte y caos.

\textbf{Lugar y fecha}: Universidad Rey Juan Carlos, 6-19 de noviembre de 2023.

\item
\textbf{Seminario:} Seminarios de Investigación del Grupo de Dinámica No Lineal, Teoría del Caos y Sistemas Complejos.

\textbf{Presentación oral}: Distancia de Hamming para medir caos en autómatas celulares elementales y juegos sociales.

\textbf{Lugar y fecha}: Universidad Rey Juan Carlos, 18 de enero de 2024.

\end{itemize}


\section*{Proyectos de investigación}

\begin{itemize}

\item
\textbf{Título}: New challenges in Nonlinear Dynamics of Complex Systems (PID2019-105554GB-I00) 

\textbf{Entidad financiadora}: Agencia Estatal de Investigación

\textbf{Duración}: 01/06/2020 - 31/05/2023

\textbf{Investigador principal}: Miguel Ángel Fernández Sanjuán

\textbf{Tipo de paticipación del doctorando}: Miembro investigador

\textbf{Cuantía de la subvención}: 84.700 euros


\item
\textbf{Título}: Explorando la dinámica no lineal de sistemas complejos. (PID2023-148160NB-I00) 

\textbf{Entidad financiadora}: Agencia Estatal de Investigación

\textbf{Duración}: 01/09/2024 - 31/08/2027

\textbf{Investigador principal}: Miguel Ángel Fernández Sanjuán

\textbf{Tipo de paticipación del doctorando}: Miembro investigador

\textbf{Cuantía de la subvención}: 87.500 euros



\end{itemize}


\section*{Becas}

\begin{itemize}


\item
\textbf{Beca}: C1PREDOC2021 Convocatoria de plazas para contratación de investigadores predoctorales en formación.\\
\textbf{Entidad financiadora}: Universidad Rey Juan Carlos.\\
\textbf{Objeto}: Contratación de personal predoctoral.\\
\textbf{Dotación económica}: 115.200 euros




\end{itemize}