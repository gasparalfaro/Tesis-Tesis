\chapter{Results and Discussion}
\label{chap:Discussion}



\begin{quotation}
	\vspace{-3cm}
    \begin{flushright}
    \begin{minipage}[t][5cm][b]{0.5\textwidth}
    {\letquote ``Strange game. The only winning move is not to play."}
    
    \bigskip
    
    -{\small  Lawrence Lasker and Walter F. Parkes, WarGames}
    \end{minipage}
    \end{flushright}
    
    \vspace{0.5cm}
\end{quotation}




We have studied a specific evolutionary game, the public goods game, and introduced modifications to examine time-dependent effects that occur in real-life scenarios. Both effects, oscillations in return efficacy and delays in punishing defectors, were found to hinder cooperation. This is because cooperators require a certain level of stability to ensure their efforts are worthwhile. Additionally, delayed punishment is less effective, as immediate consequences are more likely to achieve the intended deterrent effect.

Then we have measured complexity in the prisoner's dilemma and the public goods game with the Hamming distance metric. This has helped us understanding the chaotic pattern formations present when simulating the games spatially, with local interactions. We have obtained a measure that may correlate to the Lyapunov time. This measure tells us how much time passes until two configurations are significantly different, and also help us determining the velocity of propagation of changes in the configurations of cooperators and defectors.

Each of the analyzed games can be seen as an enormous cellular automata, and therefore, we have also analyzed the complexity of the simplest of them to understand the roots of the problem. These are the elementary cellular automata, which can be divided in $4$ classes. 

We have obtained an analogue classification to that of Wolfram, but ours focuses on the behavior of the Hamming distance of two close configurations. Class $1$ consist of cellular automata that nullifies the difference between two configurations, while in class $2$ the difference becomes two small and does not change over time. Therefore, we can say that the cellular automata in these two classes are stable. On the other hand, cellular automata in classes $3$ and $4$ are unstable, with changes rapidly propagating. In class $3$ the Hamming distance behaves chaotically representing the random like patterns that form when plotting the cells' states while in class $4$ the Hamming distance presents transient chaos, which explains the transition between complex patterns to periodical ones but with large periods characteristic of the cellular automata in this class.

By studying a control-based game, we found that success largely depends on the effective use of information. Players competing in the game have a higher chance of winning if they possess knowledge of their opponent's moves and both players' control capabilities. Additionally, a precise understanding of the system in which they are playing is crucial.

A surprising fact of the game is that the player that has the most challenging goal, to stay inside a transient chaotic region, can sometimes do so even when their control capabilities are inferior to those of their opponent.

Before studying that game, we have familiarized with the partial control, by extending the method to the control of escaping trajectories in a quick or in an orderly way. This study was also used when developing the algorithms to solve the competitive game of control. 


Through the interconnected study of game dynamics, complexity and control of chaos, we have obtained a broader understanding on how seemingly simple rules can create rich and complex behaviors. Nonetheless, through a precise understanding of the system in play, players can take advantage of the characteristics of the game dynamics and achieve their goals. Therefore, the study of game dynamics through tools that come from the complex systems domain, provides powerful insights that can be used to solve many real-world problems.

I hope that the breakthroughs and methods presented in this doctoral thesis will contribute to future research in the fields of game theory, complex systems, and control.

