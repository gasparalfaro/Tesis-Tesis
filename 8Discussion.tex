\chapter{Results and Discussion}
\label{chap:Discussion}



\begin{quotation}
	\vspace{-3cm}
    \begin{flushright}
    \begin{minipage}[t][5cm][b]{0.5\textwidth}
    {\letquote ``Strange game. The only winning move is not to play."}
    
    \bigskip
    
    -{\small  Lawrence Lasker and Walter F. Parkes, WarGames}
    \end{minipage}
    \end{flushright}
    
    \vspace{0.5cm}
\end{quotation}





We have studied one particular evolutionary games, the public goods game, and made alterations to it to examine time-dependent effects that happen in real-life situations. Both examined effects, the oscillations in returns efficacy and a delay in punishment to defectors, hindered cooperation. This is because cooperators need a certain stability to know that their efforts will be worthwhile, and in the case of a delay in punishment, because a rapid punishment is always more susceptible to make the intended effect.

Then we measured complexity in the prisoner's dilemma and the public goods game with the Hamming distance metric. This helped us understand the chaotic pattern formations present when simulating the games spatially, with local interactions. We obtained a measure that may correlate to the Lyapunov time. This measure tells us how much time passes until two configurations are significantly different, and also help us determine the velocity of propagation of changes in configurations of cooperators and defectors.

The studied games can be seen as enormous cellular automata, and therefore, we also analyzed the complexity of the most simple of them to understand the roots of the problem. This are the elementary cellular automata, which can be divided in $4$ classes. 

We obtained an analogue classification to that of Wolfram, but ours focuses on the behavior of the Hamming distance of two close configurations. Class $1$ consist of cellular automata that nullifies the difference between two configurations, while in class $2$ the difference becomes two small and does not change over time. Therefore we can say that the cellular automata in these two classes are stable. On the other hand cellular automata in classes $3$ and $4$ are unstable, with changes rapidly propagating. In class $3$ the Hamming distance behaves chaotically representing the random like patterns that form when plotting the cells' states while in class $4$ the Hamming distance presents transient chaos, which explains the transition between complex patterns to periodical ones but with large periods characteristic of the cellular automata in this class.

By studying the a game of control, we obtained that the key to success can be the correct use of information. The player's that compete in the game are more likely to win the game if they have information about their opponent's moves and of the capabilities of control of both. Is also key to have a precise understanding of the system in which they are playing.


A surprising fact of the game is that the player that has the most challenging goal, to stay inside a transient chaotic region, can sometimes do so even when their control capabilities are inferior to their those of their opponent.

Before studying that game we familiarized with the partial control, by extending the method to the control of escaping trajectories in a quick or in an orderly way. This study was also used when developing the algorithms to solve the competitive game of control. 


Through the interconnected study of game dynamics, complexity and control of chaos, we obtained a broader understanding on how seemingly simple rules can create rich and complex behaviors. Nonetheless, through a precise understanding of the system in play, players can take advantage of the characteristics of the game dynamics and achieve their goals. Therefore the study of game dynamics through tools that come from the complex systems field, provides powerful insight that can be used to solve many real-world problems.

I hope that the breakthroughs and methods collected in the study of this doctoral thesis will help in the investigation of following research in the fields of game theory, complex systems and control.

