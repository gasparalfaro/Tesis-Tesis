\chapter*{Preface}


\begin{quotation}

\begin{flushright}
\begin{minipage}[t][5cm][b]{0.5\textwidth}
{\letquote ```I checked it very thoroughly,' said the computer, `and that quite definitely is the answer. I think the
problem, to be quite honest with you, is that you've never actually known what the question is.'"}

\bigskip

-{\small  Douglas Adams, The Hitchhiker’s Guide to the Galaxy}
\end{minipage}
\end{flushright}

\vspace{0.5cm}
\end{quotation}

%%%%%%%%%%%%%%%%%%%%%%%%%%% Copiado de Julia
This doctoral thesis is the result of $4$ years of work within the Research Group on Nonlinear Dynamics, Chaos and Complex Systems of the Rey Juan Carlos University.

Beginning with an introductory chapter presenting the basic ideas, the results are afterwards organized in $5$ chapters which correspond to different scientific publications. At the beginning of this thesis, we studied the public goods game on a square lattice with imitation as the evolutionary drive where players chose three strategies: cooperate, defect, or punish the defectors. After some study on the result of time-dependent effects on the game, which we collect in Chapter~\ref{chap:TimeEffects}, we began to question how to know and measure whether the dynamics of the game were complex. In Chapter~\ref{chap:HammingGames} we answer this question and in Chapter~\ref{chap:HammingECA} we follow up the question to Evolutionary Cellular Automata. 

As a result of the studies in the end of master dissertation, we published an article where we used partial control to force orbits escape from a transient chaotic region. This study is summarized in Chapter~\ref{chap:ForcingEscape}. At the last year of the doctorate studies we wanted to bridge both subjects of study with a final research. In Chapter~\ref{chap:PartialControlGame} we use partial control towards the advantage of players in a game. This closed the cycle of investigation that begun with the master's thesis and continued in the doctorate research. Next we give the structure of the thesis and a summary of each chapter.



%%%%%%%%%%%%%%%%%%%%%%%%%%%


\vspace{1cm}

{\bf  Chapter 1. Introduction}

\vspace{0.6cm}

{\bf  Chapter 2. Methodology}

\vspace{0.6cm}

{\bf  Chapter 3. Time-dependent effects on the public goods game}  
\vspace{0.6cm}

Here we study two different time-depending effects. We analyze the effects of perturbations in the main parameter controlling the payoff of the players of the public goods game. We also studied a delay on the time it takes for the punish to affect defectors. The main result were that both the oscillation in the parameter and the delay in punishment hindered cooperation.
 
\vspace{0.6cm}

\vspace{0.6cm}

{\bf  Chapter 4. Hamming distance as a measure of spatial chaos in evolutionary games}  
\vspace{0.6cm}

In this chapter we studied the complexity of two relevant social games. For this we measured the Hamming distance of two configurations varying in just one agent. We approached first the prisoner dilemma since a previous study by Martin Nowak and Robert May hinted at a complex behaviour in a specific parameter regime. We found that under this parameter regime, the distance diverges towards saturation, and is null for the rest of parameter values. This granted us confidence in the algorithm and used it on the public goods game, where we found that, when both cooperators and defectors are present, even when the dynamics do not seem that complex. Although the divergence starts earlier for the cases where the dynamics is more complex.



\vspace{0.6cm}

{\bf  Chapter 5. Classification of Cellular Automata based on Hamming distance}

\vspace{0.6cm}

Following the study of complexity on the previous games, we study here the complexity of Elementary Cellular Automata. We measure the Hamming distance from two configurations that differ only at one cell at the beginning and analyze the temporal behaviour of the distance. Depending on the distance periodicity or chaoticity, we classify each rule and find that the classification matches Wolfram's own classification.

\vspace{0.6cm}

{\bf  Chapter 6. Forcing the escape of orbits from a transient chaotic region with partial control} 

\vspace{0.6cm}

Here we shift our attention to partial control. We develop here a method for controlling a orbit inside a chaotic transient to escape from it and go directly to one of its atractors. We develop two methods: one to escape the quickest way possibly and another to escape at an exact given number of iterations.

\vspace{0.6cm}


%TODO Título
{\bf  Chapter 7. Two-player Yorke's game of survival in chaotic transients.}

\vspace{0.6cm}

Using techniques from partial control we aim to design a game where two players compete in a region with transient chaos. By defining and computing the victory sets we get the set of initial conditions that grants the victory to either player. There are cases with regions where victory is not granted for any of them, so we design the best strategy for each player.


\vspace{0.6cm}

{\bf  Chapter 8. Conclusions}

\vspace{0.6cm}

Finally, the main conclusions of this thesis are summarized.