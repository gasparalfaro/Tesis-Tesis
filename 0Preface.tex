\chapter*{Preface}


\begin{quotation}

\begin{flushright}
\begin{minipage}[t][5cm][b]{0.5\textwidth}
{\letquote ```I checked it very thoroughly,' said the computer, `and that quite definitely is the answer. I think the
problem, to be quite honest with you, is that you've never actually known what the question is.'"}

\bigskip

-{\small  Douglas Adams, The Hitchhiker’s Guide to the Galaxy}
\end{minipage}
\end{flushright}

\vspace{0.5cm}
\end{quotation}


This doctoral thesis is the result of four years of work within the Research Group on Nonlinear Dynamics, Chaos and Complex Systems of the Rey Juan Carlos University.

Beginning with an introductory chapter presenting the basic ideas, the results are organized in five chapters which correspond to different scientific publications. In Chapter~\ref{chap:TimeEffects}, the study of the public goods game with a square lattice and imitation as the evolutionary drive is included, where players chose three strategies: cooperate, defect, or punish the defectors. After an analysis of time-dependent effects on the game, we have questioned ourselves several issues concerning the complexity of the local dynamics of the game. In Chapter~\ref{chap:HammingGames}, we have answered this issue and in Chapter~\ref{chap:HammingECA}, we have followed up the question to Elementary Cellular Automata. 

Another fundamental aspect of this doctoral thesis is the concept of control. First, we employ a method known as partial control to prevent orbits from escaping a transient chaotic region in nonlinear dynamics. This investigation is detailed in Chapter~\ref{chap:ForcingEscape}. Later, we extend this idea to a novel problem in game theory, establishing a connection between control theory and strategic decision-making. The results of this research are presented in Chapter~\ref{chap:PartialControlGame}, where partial control principles are applied to construct and analyze a competitive game between two players. Finally, we outline the structure of the thesis and provide a summary of each chapter.




%\vspace{1cm}

\clearpage

{\bf  Chapter 1. Introduction}

\vspace{0.6cm}

To introduce the topics of this thesis, we first give a few basics ideas of game theory, complex systems, and partial control method of controlling chaos along a review of the investigations done in the doctoral thesis.

\vspace{0.6cm}

{\bf  Chapter 2. Methodology}

\vspace{0.6cm}

Next, to familiarize the reader with the tools and methods used in this research, we provide an explanation of their application. The most important among them are numerical simulations of agent-based models and cellular automata. Additionally, we describe the methods used to compute Lyapunov exponents and construct bifurcation diagrams.


\vspace{0.6cm}

{\bf  Chapter 3. Time-dependent effects on the public goods game}  

\vspace{0.6cm}

Here, we examine two distinct time-dependent effects on the public goods game with punishment. First, we analyze the impact of perturbations in the key parameter that governs the players' payoffs. Additionally, we investigate the effect of introducing a delay in the time it takes for punishment to influence defectors. Our main finding is that both parameter oscillations and delayed punishment hinder cooperation.


\vspace{0.6cm}

{\bf  Chapter 4. Measuring complexity in evolutionary games with the Hamming distance} 
 
\vspace{0.6cm}


In this chapter, we analyze the complexity of two significant social games. To this end, we measure the Hamming distance between two configurations differing by a single agent. We begin with the prisoner's dilemma, where our analysis confirms the presence of spatiotemporal chaos in certain parameter regimes, consistent with previous findings by May and Nowak. We then examine the public goods game, but the results remain inconclusive due to the inherent randomness of the evolutionary model used.


\vspace{0.6cm}

{\bf  Chapter 5. Classification of Cellular Automata based on Hamming distance}

\vspace{0.6cm}

Building on the complexity analysis of the previous games, we now examine the complexity of Elementary Cellular Automata. We measure the Hamming distance between two initial configurations differing by a single cell and analyze its temporal evolution. Based on whether the distance exhibits periodic or chaotic behavior, we classify each rule and find that our classification aligns with Wolfram's established framework.

\vspace{0.6cm}

{\bf  Chapter 6. Escaping from transient chaos with partial control} 

\vspace{0.6cm}

Here, we turn our focus to partial control. We develop a method for guiding a trajectory within a chaotic transient to escape directly to one of its attractors. Specifically, we propose two approaches: one that ensures the fastest possible escape and another that allows for escape after an exact predetermined number of iterations.

\vspace{0.6cm}


{\bf  Chapter 7. Two-player Yorke's game of survival in chaotic transients.}

\vspace{0.6cm}

Using techniques from partial control, we design a game in which two players compete within a region of transient chaos. By defining and computing the winning sets, we determine the initial conditions that guarantee victory for each player. Furthermore, we analyze how the winning conditions change based on the information available to each player regarding their opponent's actions.


\vspace{0.6cm}

{\bf  Chapter 8. Results and Discussion}

\vspace{0.6cm}

The main results of the investigations are presented and discussed.


\vspace{0.6cm}

{\bf  Chapter 9. Conclusions}

\vspace{0.6cm}

Finally, the main conclusions are summarized.