\chapter{Methodology}
\label{chap:Method}


\begin{quotation}


	\vspace{-3cm}


    \begin{flushright}
    \begin{minipage}[t][5cm][b]{0.5\textwidth}
    {\letquote ``The purpose of computing is insight, not numbers."}
    
    \bigskip
    
    -{\small  Richard Hamming}
    \end{minipage}
    \end{flushright}
    
    \vspace{0.5cm}
\end{quotation}





For this doctoral thesis, we employ a theoretical and computational approach, employing a combination of analytical methods and numerical simulations to corroborate theoretical predictions with observable phenomena. These simulations provided us with insights and a better understanding of the problems at hand. Following, we describe the methods applied.


\section{Monte Carlo agent-based simulations}

For Chapters~\ref{chap:TimeEffects} and~\ref{chap:HammingGames} we make agent-based simulations of individuals playing social games. Each agent has a strategy to play the game. Games are played between neighbors in a square grid and gaining different payoff, which represent the fitness of each player. After each game has been played and we have obtained the payoffs, through the Monte Carlo method~\cite{MonteCarlo}, random individuals are selected to update their strategy to that of a random neighbor. The likelihood of adopting the new strategy will depend on the difference of both agents' payoff. The games are iterated many times and we can see the dynamical changes of strategies by plotting each strategy in different colors. These simulations were produced using custom made code in the \textit{Julia} programming language.

\section{Numerical analysis tools}

This collects methods to analyze systems and data to characterize nonlinear systems. This methods include elaboration of bifurcation diagrams, calculations of Lyapunov exponents, and the algorithm used in Chapters~\ref{chap:HammingGames} and~\ref{chap:HammingECA} to asses complexity using the Hamming distance.

\subsection{Bifurcation diagrams}

A bifurcation diagram~\cite{Bifurcation} is a graphic that represents the different dynamical behaviors in the systems dynamics according to the variation of a certain parameter. The graphic plots all the values of the steady state through time after relaxation versus each value of the parameter. The bifurcation diagram shows a single line when the system is in a fixed state. As the parameter changes, the line may bifurcate in two or more lines as the system's fixed state becomes a periodic state. For system that present nonlinear behavior, as the parameter reaches a critical value, the period increases by doubling, until the critical value is reached and a continuous set of points is represented in the bifurcation diagram for each parameter value. This indicates that the system has a chaotic atractor and the dynamics is chaotic.

To obtain the bifurcation diagram one must first let the dynamics get to a steady state where the transient dynamics is discarded. In the case of multistable systems, one must calculate the steady state for more than one initial condition, wich produces different steady states because they reach different atractors. By plotting the steady state of each initial condition separately or in a different color, we can analyze the differences. 

A bifurcation diagram for a multistable system was plotted in Chapter~\ref{chap:ForcingEscape}. It was computed using \textit{MATLAB}. 

\subsection{Lyapunov exponents}

To know when the dynamics of a system is chaotic, one can measure the local divergence of trajectories. One method to do this is to calculate the Lyapunov exponents~\cite{LyaExp}. There are as many Lyapunov exponents as the dimensionality of phase space. In chaotic systems at least one of the Lyapunov exponents is positive, so one could simply calculate the Maximum Lyapunov Exponent MLE to determine the complexity of the systems. For discrete time systems, the MLE can be obtained with the following formula:
\begin{equation}
\lambda(x_0) = \lim_{n \to \infty}\dfrac{1}{n}\sum_{i=0}^{n-1}\log|\dfrac{df(x_i)}{dt}.
\end{equation}
This formula was used in Chapter~\ref{chap:ForcingEscape} implemented using \textit{MATLAB}.

\subsection{Hamming distance divergence measure}

The Hamming distance measures the number of different elements in a group, vector or matrix. To measure the divergence of similar initial conditions in Chapters~\ref{chap:HammingGames} and~\ref{chap:HammingECA}, we first let the system get to a steady state. Then, we made a copy of the system and changed only one agent state. After this, we let the systems evolve and measure the Hamming distance between the two configurations. If the Hamming distance grows, this is a sign that the system may be chaotic (or random). Since the size of the system is finite, the Hamming distance saturates at a certain point which can be calculated. It depends on the proportions of the fixed-state frequencies of each strategy.

In Chapter~\ref{chap:HammingECA} we found that the different cellular automata Wolfram classes can be obtained by looking at the Hamming distance divergence, whether it is zero, or grows, and depending on the behavior when it saturates.



\begin{thebibliography}{02}


\bibitem{MonteCarlo}
D. Reiter. The Monte Carlo method, an introduction. In \textit{Computational Many-Particle Physics}
Lecture Notes in  Physics \textbf{739}, 63–-78 (Springer, Berlin, Heidelberg, 2008).
\url{https://doi.org/10.1007/978-3-540-74686-7_3}


\bibitem{Bifurcation}
H. E. Nusse, J. A. Yorke, E. J. Kostelich, 
Bifurcation Diagrams. In \textit{Dynamics: Numerical Explorations},
Applied Mathematical Sciences \textbf{101} \textbf{101} (Springer, New York, NY 1994).
\url{https://doi.org/10.1007/978-1-4684-0231-5_6}


\bibitem{LyaExp}
A. Pikovsky and A. Politi, 
\textit{Lyapunov Exponents: A Tool to Explore Complex Dynamics},
(Cambridge University Press, 2016).




\end{thebibliography}
