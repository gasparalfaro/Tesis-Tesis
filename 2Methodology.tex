\chapter{Methodology}
\label{chap:Method}


\begin{quotation}


	\vspace{-3cm}


    \begin{flushright}
    \begin{minipage}[t][5cm][b]{0.5\textwidth}
    {\letquote ``The purpose of computing is insight, not numbers."}
    
    \bigskip
    
    -{\small  Richard Hamming}
    \end{minipage}
    \end{flushright}
    
    \vspace{0.5cm}
\end{quotation}


At the beginning of this doctoral thesis, a thorough bibliographic research was done in order to know the state of the art on the field of evolutionary dynamics and control of chaos. At the time I was comfortable with the elements of the subjects, I began to replicate the results of the paper \cite{Replicate}. The study made numerical simulations of the public goods game using code, which I successfully replicated from scratch. Then we began to introduce variations to answer new questions that my co-directors and I came out.


Throughout this work, we employ a combination of analytical methods and numerical simulations to bridge the gap between theoretical predictions and observable phenomena. This simulations provided us with insights and a better understanding of the problems.

As we got results for the different questions we asked ourselves, we noted them down and published them in journals of high impact in the complex systems field, like Chaos, Nonlinear Dynamics, and Physical Review E.

Finally all main results were collected in this thesis distributed among chapters that correspond to $5$ papers published, with myself as the first author.

\begin{thebibliography}{02}


\bibitem{Replicate}
P. Zhu,H. Guo, H. Zhang, Y. Han, Z. Wang, and C. Chu
The role of punishment in the spatial public goods game,
Nonlinear Dyn \textbf{102}, 2959--2968 (2020) 
\url{https://doi.org/10.1007/s11071-020-05965-0}




\end{thebibliography}
