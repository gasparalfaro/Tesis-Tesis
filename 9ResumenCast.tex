\chapter{Resumen en castellano}



\begin{quotation}
	\vspace{-3cm}
    \begin{flushright}
    \begin{minipage}[t][5cm][b]{0.5\textwidth}
    {\letquote ``Que la realidad te parezca absurda es la mayor crítica que le puedes hacer"}
    
    \bigskip
    
    -{\small  Laura Fernandez}
    \end{minipage}
    \end{flushright}
    
    \vspace{0.5cm}
\end{quotation}






Esta tesis trata sobre juegos evolutivos, complejidad y control del caos. Estos temas, que pueden parecer dispares, han sido interconectados en los diferentes capítulos presentes aquí. 

Al principio de la tesis estudié dos juegos sociales. El juego de los bienes públicos fue analizado al introducir componentes dependientes del tiempo, y se observó cómo afectaban negativamente a la cooperación entre individuos. Después se analizó la complejidad de este, el juego de los bienes públicos, y del dilema del prisionero. Ambos resultaron presentar complejidad tras el análisis de la divergencia entre configuraciones muy parecidas al inicio. La distancia de Hamming entre las configuraciones crecía rápidamente.

Más tarde se analizaron de la misma manera todos los automatas celulares elementales y se elaboró una clasificación análoga a la de Wolfram dependiente del comportamiento de la distancia de Hamming. Se destaca el comportamiento caótico transitorio de la clase $4$, que se podría relacionar con el fenómeno de \textit{edge of chaos} (límite del caos).

En cuanto a control del caos se extendió el estudio del control parcial de manera que el controlador pudiera arrancar a las trajectorias que presentan caos transitorio de la región transitoria de la manera más rápida posible o de una manera contgrolada. Cabe destacar que se pudo realizar este control con unas cotas menores que las del ruido presente en el sistema.

Por último se creo y resolvió un juego de supervivencia entre dos jugadores enfrentados entre sí en sistemas con caos transitorio con la ayuda del control parcial. Se obtuvieron las condiciones iniciales para las que cada jugador tiene asegurada la victoria. Se observó que estas condiciones son más amplias para aquellos jugadores que tienen información de los movimientos del oponente.

Con todo esto se concluye que el estudio de la complejidad en juegos y el análisis de los sistemas con herramientas del campo de sistemas complejos ayudan a resolver problemas o a tener ventaja en determinados juegos.


